% Note: this file can be compiled on its own, but is also included by
% diss.tex (using the docmute.sty package to ignore the preamble)
\documentclass[12pt,a4paper,twoside]{article}
\usepackage[pdfborder={0 0 0}]{hyperref}
\usepackage[margin=25mm]{geometry}
\usepackage{graphicx}
\usepackage{parskip}
\hypersetup{
    colorlinks=false,
    linkcolor=blue,
    filecolor=magenta,      
    urlcolor=blue,
}
\begin{document}

\begin{center}
\Large
Computer Science Tripos -- Part II -- Project Proposal\\[4mm]
\LARGE
QWOP in JavaScript \\[4mm]

\large
George Andersen, gla23, Robinson College

\today

\end{center}

\vspace{5mm}

\textbf{Project Supervisor:} Dr A.~Beresford

\textbf{Director of Studies:} Prof A.~Mycroft

\textbf{Project Overseers:} Prof R.~Anderson \& Prof J.~Bacon

% Main document

\section*{Introduction}

% QWOP Description
QWOP is a 2008 ragdoll-based browser video game created by Bennett Foddy. Players control an athlete using only the Q, W, O, and P keys. The aim of the game is to complete a 100m race without falling over, which is a lot harder than it first seems. 
Once the athlete reaches the end of the 100m, the player is given the time they did it in. If a player can complete the 100m, they can then try to do it as quickly as possible. After the game went viral in 2010, people have been trying to break the world record, one speed run video attempting this on YouTube has 2.5 million views.
%The game helped Foddy's site reach 30 million hits.

QWOP uses a ragdoll physics simulation for the athlete. Each part of the runner's body is a rigid body with its own velocity, rotation and other physical quantities. It has joint constraints with connecting rigid bodies which allows the body to move without ending up in improbable or impossible positions.
The simulation also emulates muscles that can be controlled with the Q, W, O and P keys, so that the user can make the runner move. The Q key drives the runner's right thigh forward and left thigh backward, and the W key does the opposite. The O and P keys do the same but for the calves.
Though the objective of QWOP is simple, the game, ever since it was released, has been notorious for being difficult to master due to its controls with the Q, W, O and P keys.

 %Look up the correct terminology for this kind of system. Clearly you need to model some of the physics, but you perhaps also need to look up "forward kinematics", "inverse kinematics" and similar. An hour or two on Wikipedia now will help you work out the basics and how to describe your approach more precisely.

% Outline of plans (remake using JS + Phaser)
In this project I will remake QWOP in JavaScript using Phaser. Phaser has a physics system called P2 that can be used to recreate the physics behind QWOP.

Phaser has objects called \emph{sprites} which can be used to build the game. Each of these can be attached to a P2 physics body used in Phaser's P2 physics system. I can then build on this system to give these bodies the correct physical properties.
More details on how this will work can be found in the work to be done section.


%After making the physics simulation I would need to add graphics to the joints to make it look as well as run like a human. Another thing that would need to be added is a system to collision check when the man falls over, and measure how far the player got from the start. Then have it show your score and keep high scores etc just like in the real qwop.
% Deterministic scoring, based on your later comments, sounds like something thing you would want to support.
 

\section*{Starting point}

%\emph{Describe existing state of the art, previous work in this area, libraries and databases to be used. Describe the state of any existing codebase that is to be built on.}

One major codebase the project will be using is Phaser, a JavaScript library designed to make games.
So far I have downloaded Phaser and used some of the examples to explore the physics engine.

I have also spent time familiarising myself with the tools I will be using for the project and working out a work flow. These include Git, GitHub, WampServer, Phaser and Sublime text with LaTeXTools.

\section*{Resources required}

%\emph{A note of the resources required and confirmation of access.}

For this project I shall use my own laptop running Windows.
My Laptop's specifications:
Intel i7 16GB RAM, 1 TB+8 GB SSHD, AMD R9 GPU.

If my laptop fails I can continue development on the MCS machines in the Computer Lab.

I will backup my code and the writeup on a private repository on GitHub.

I require no other special resources.

\section*{Work to be done}

%\emph{Describe the technical work.}

The project breaks down into the following sub-projects:

Each of these sections include researching how the task can be completed in the Phaser physics engine, and then creating it in the project.

\begin{enumerate}
\item Create a ragdoll model of the runner's body out of Phaser sprites with correct distance constraints between sections so that it keeps together whilst collapsing to the floor.
\item Ensure the body has inertia, give the floor friction and apply other physics features so it acts like a runner rather than collapsing on itself.
\item Add constraints on how far a joint can rotate. Add these constraints on the model of the athlete so the runner has human like joint constraints.
\item Add graphics to the body; give each sprite joint an image so together they look like an athlete's body.
\item Make the athlete controllable using the Q, W, O and P keys. Receive this input into the game, and give the correct rotational force to each section of the body as in the original game.
\item Calculate how far the runner has moved from the start and determine the user's score. Display this on the screen.
\item Add collision detection with the floor for the parts of the body that touch the floor when the runner falls over. Restart the game when the athlete falls over.
\item Check whether the runner has finished the race by checking whether the distance travelled is over 100m, and if so, record the time it took to complete the race.
\end{enumerate}

\section*{Success criteria}

%\emph{Describe what you expect to be able to demonstrate at the end of the project and how you are going to evaluate your achievement.}

%How will you evaluate the project? How can you determine "success"? Do you need to do a user study? Can you do something else to determine whether it "works"? Perhaps you can compare the output from a particular key combination in the original and your new version?

%TODO: user study? and optionally deterministic.
The project will be a success if it works in a similar way to the original QWOP. This can be quantified by checking the athlete has the same physical qualities that make it seem human, and the same gameplay items that make the game work. To do this I can check each section of the work to be done section has been completed.

if have enough time for extension

Another thing that will make the project a success is that the game is deterministic; gives the same score each time a certain key combination is entered. I want to support this as it will be helpful for an extension idea that is explained in the possible extensions section.
To test this I can code an input into the game that can be repeated, rather than use input from the keyboard. If the game is deterministic, it will give the same score each time. A different method I could use is to make an AutoHotkey script that gives an input, and it can be run multiple times to check the score is the same each time.

Another evaluation metric I can use is does the athlete respond in similar ways in both versions to the same input key combinations? This could be checked by programming the same input into both the original and my version, and checking the athlete goes through the same type of movement. 
As the original QWOP isn't deterministic, and I plan for my version to be, the movement is unlikely to match exactly. This test however would still be useful, as it can check the movements are similar.
Since I will not have access to the internal workings of the original QWOP game, and therefore cannot code the input directly, It would be better to use the AutoHotkey script method mentioned earlier to give the desired consistent input.
For example if q and p are pressed for 500 milliseconds then swap to w and o, swapping back and forth every 500 milliseconds, does the athlete move in the same way?

\section*{Possible extensions}

%{\em Potential further envisaged evaluation metrics or extensions.} add deterministic here

If I achieve my main result and still have spare time I shall try an experiment of creating an AI module that attempts to play QWOP. It could be broken down into these sections:

\begin{enumerate}
\item Design an encoding scheme (or multiple to see which works best) that can be used to control input to the game. Each instantiated value of the scheme (runner) can be used to control the input to a game. A deterministic version of the game will give the same score each time the same runner plays a game.

\item Provide the AI module with access to the rotations of the different joints whilst the athlete is running so that the AI can use this information.

\item Test runners giving them a fitness score of how far they managed to run before falling over.
  
\item Make a genetic algorithm that can repeatedly test runners, and mutate them with the effect of increasing the fitness of the runners over time.
\end{enumerate}

 
% Old version of explanation
%An extension goal is to make an ai module that would give input to the game with the aim of getting to the end of the race as fast as possible. It would take an output from the game of the rotations of the joints and other data needed to make the ai work, and would then give an input to the game that should be intelligent enough to run the 100m.


\href{https://dl.acm.org/citation.cfm?id=2598248}{This Paper} \cite{ray2014evolving} has been written about evolving QWOP gaits.
The writer of this paper tests various genetic algorithms to evolve a gait (walking motion) for the original QWOP.
To do this he uses a similar approach to the one I have described above. The difference is that since he is using the original game, there is no access to the internal workings of the game, and therefore can only get information from the game using a screen capture.
He uses this to work out the current score, and this is the only information he uses to evolve the runners. Since I am making my own version, I have access to the data in the game, and so I could use this information directly.

I could trying reimplementing some of the coding schemes or genetic algorithms the above paper uses, but I will have more options as I have the extra data to use.

%TODO: fix this -with deterministic stuff
The paper found that the original qwop wasn't deterministic with the inputs given to it, so one input to the game could give different scores on different tries. This made it harder for the genetic algorithms to be effective. 
The problem was that when it would pick the runners as they got a higher score, this could be because they happened to get a higher score that time, rather than being an inherintely better runner.
If I get on to the extension, it will be to my benefit that I am remaking the game as I aim for it to be deterministic.


%Another possible extension is making another model that isn't a human. %suggested by alan

\section*{Timetable}
%\emph {A workplan of perhaps ten or so two-week work-packages, as well as milestones to be achieved along the way. Provide a target date for each milestone.}

%Planned starting date is \today.

% You need to put down what you will do for each fortnight between the start of the project and the hand-in date. Please list specific dates. You should schedule a week off for Christmas. As a rule of thumb, you need to schedule some time for research in October; have the core up and running in January so that you can present it working at the overseer review; perform the evaluation in February and early March, perhaps intermixed with revisions and improvements. Schedule two weeks in early March for "contingency time / extensions"; the rest of the time should be left for write-up.


\begin{enumerate}
%Michaelmas weeks 3 and 4: 
\item \textbf{19th October-1st November} After handing in project proposal, I can start researching how exactly to get the physics engine to do what I want.

%Michaelmas weeks 5 and 6:
\item \textbf{2nd - 15th November} Start implementation of the model of runner -- go through first steps in the Work to be done section.

%Michaelmas weeks 7 and 8: 
\item \textbf{16th - 30th November}
Add more of the physics features, work out how to apply the joint restrictions, add these to the model etc

%Michaelmas vacation
\item \textbf{1st - 14th December}
Student getaway with church, travel home and visit my sister in Italy with parents
\item \textbf{15th - 21st December}
Get core up and running, work on how the program will work as a whole with keeping score etc.
\item \textbf{22th - 28th December}
Week off for Christmas
\item \textbf{29th December - 8th January}

\item \textbf{9th - 14th January}
Ciccu getaway and move back to Cambridge

%Lent weeks 1 and 2: 
\item \textbf{18th - 31st January}
Make sure all the sections in work to be done are complete so I can present it working at the overseer review.
Write progress report so that it can be handed in on the 2nd Feb. 

%Lent weeks 3 and 4: 
\item \textbf{1st - 14th February}
Start evaluation of project

%Lent weeks 5 and 6: 
\item \textbf{15th - 28th February}
Give time for revisions and improvements of project, whilst finishing evaluation of the project.

%Lent weeks 7 and 8: 
\item \textbf{1st - 14th March}
Contingency time/extensions

%Easter vacation:
\item \textbf{15 March - 25th April}
Write dissertation main chapters.

%Easter weeks 1 and 2: 
\item \textbf{26th April - 9th April}
Further evaluation and complete dissertation.

%Easter term week 3:
\item \textbf{10th - 17th May}
Proof reading and then an early submission so as to concentrate on examination revision.

\end{enumerate}

\bibliographystyle{plain}
\bibliography{proposal} 

\end{document}
