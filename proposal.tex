% Note: this file can be compiled on its own, but is also included by
% diss.tex (using the docmute.sty package to ignore the preamble)
\documentclass[12pt,a4paper,twoside]{article}
\usepackage[pdfborder={0 0 0}]{hyperref}
\usepackage[margin=25mm]{geometry}
\usepackage{graphicx}
\usepackage{parskip}
\begin{document}

\begin{center}
\Large
Computer Science Tripos -- Part II -- Project Proposal\\[4mm]
\LARGE
QWOP in JavaScript \\[4mm]

\large
George Andersen, Robinson College

\today

\end{center}

\vspace{5mm}

\textbf{Project Supervisor:} Dr A.~Beresford

\textbf{Director of Studies:} Dr A.~Mycroft

\textbf{Project Overseers:} Dr R.~Anderson \& Dr J.~Bacon

% Main document

\section*{Introduction}

\emph{The problem to be addressed.}
 
 % qwop description
 The simulation should have something that emulates muscles that can be controlled with the qwop keys, so that the user can make the runner move. The Q key should drive the runner's right thigh forward and left thigh backward, and the W key should do the opposite. The O and P keys should do the same but for the calves.

 %Look up the correct terminology for this kind of system. Clearly you need to model some of the physics, but you perhaps also need to look up "forward kinematics", "inverse kinematics" and similar. An hour or two on Wikipedia now will help you work out the basics and how to describe your approach more precisely.

% outline what your plans are (remake using JS + Phaser)
 The project proposed is to remake qwop using JavaScript. This would mean using Phaser and it's physics system to simulate a body that acts like that of a human. The method used would be to have some of Phasers 'sprite's that are at different joints in the body that have distance constraints with other sprites so that they stay the same distance relative to each other like each end of an arm would. There would need to be other constraints on how far round an arm can bend round etc.


 After making the physics simulation I would need to add graphics to the joints to make it look as well as run like a human. Another thing that would need to be added is a system to collision check when the man falls over, and measure how far the player got from the start. Then have i  show your score and keep high scores etc.
% Deterministic scoring, based on your later comments, sounds like something thing you would want to support.
 

 An extension goal is to make an ai module that would give input to the game with the aim of getting to the end of the race as fast as possible. It would take an output from the game of the rotations of the joints and other data needed to make the ai work, and would then give an input to the game that should be intelligent enough to run the 100m.
 https://dl.acm.org/citation.cfm?id=2598248 This paper had been written about evolving qwop gates using various genetic models.

reimplement paper stuff?

One thing that it found is that the original qwop wasn't deterministic with the inputs given to it, so one input to the game could give different scores on different tries. This made making the genetic algorithms work a lot harder as it would pick and choose different control methods as they got a higher score, but the problem was that this could be because they happened to get a higher score that time, rather than being an inherintely better control. If I get on to the extension, it will be to my benefit that as I am remaking the game and it can be made deterministic.

\section*{Starting point}

\emph{Describe existing state of the art, previous work in this area, libraries and databases to be used. Describe the state of any existing codebase that is to be built on.}

One major codebase the project will be using is Phaser, a JavaScript library designed to help make games. So far I have downloaded phaser and used some of the examples to mess around with it's physics engine.

\section*{Resources required}

\emph{A note of the resources required and confirmation of access.}

For this project I shall mainly use my own laptop that runs Windows. I will backup my code and the writeup on a private repository on GitHub, incase the laptop it is stored on laptop fails. I require no other special resources.

\section*{Work to be done}

\emph{Describe the technical work.}

The project breaks down into the following sub-projects:

\begin{enumerate}

% todo: neeten up and increase detail
Major work items/substance: core/extension


\item c1 - use the physics system to create the body parts put together with correct constraints on distances and rotations etc also give it the right inertia gravity and friction, all such that it acts like a body.

\item c2 - adding graphics to the body that works with the body - correct rotations with the physics etc

\item c3 - Make the runner controllable by the qwop keys. Add system to make it work out and show how far the runner has got. Add collision detection with the floor for the parts of the body that touch the floor when the runner falls over. Restart and check score when he falls over. Check whether the runner has finished the race, and record time of run.


\end{enumerate}

\section*{Success citeria}

\emph{Describe what you expect to be able to demonstrate at the end of the project and how you are going to evaluate your achievement.}

%How will you evaluate the project? How can you determine "success"? Do you need to do a user study? Can you do something else to determine whether it "works"? Perhaps you can compare the output from a particular key combination in the original and your new version?

The project will be a success if\ldots maybe done c1-whatever


\section*{Possible extensions}

{\em Potential further envisaged evaluation metrics or extensions.}

If I achieve my main result early I shall try an experiment of creating an ai module that attempts to run the qwop man. It could be broken down into these sections:

e1 - make an encoding scheme (or multiple to see which works best) that can be used to control input to the game. Each instantiated value of the scheme (runner) can be used to control the input to the game, and will give the same score each time, given the game is deterministic.

e2 - make a way for an ai module to be able to access the rotations of the different joints whilst it is running so that it can be more inteligent by using that information to run.

e3 - Make a method that is able to test runners giving them a fitness score of how far they managed to run before falling over.

e4 - Make a genetic algorithm that can repeatedly test runners, and mutate them with the effect of increasing the fitness of the runners over time.


\section*{Timetable} \emph {A workplan of perhaps ten or so two-week work-packages, as well as milestones to be achieved along the way. Provide a target date for each milestone.}

Planned starting date is \today.

You need to put down what you will do for each fortnight between the start of the project and the hand-in date. Please list specific dates. You should schedule a week off for Christmas. As a rule of thumb, you need to schedule some time for research in October; have the core up and running in January so that you can present it working at the overseer review; perform the evaluation in February and early March, perhaps intermixed with revisions and improvements. Schedule two weeks in early March for "contingency time / extensions"; the rest of the time should be left for write-up.

\begin{enumerate}

\item \textbf{Michaelmas weeks 2--4} Learn to use X. Read book Y. Read papers Z.

\item \textbf{Michaelmas weeks 5--6} Do preliminary test of Q.

\item \textbf{Michaelmas weeks 7--8} Start implementation of main task A.

\item \textbf{Michaelmas vacation} Finish A and start main task B.

\item \textbf{Lent weeks 0--2} Write progress report. Generate corpus of
  test examples. Finish task B.

\item \textbf{Lent weeks 3--5} Run main experiments and achieve working project.

\item \textbf{Lent weeks 6--8} Second main deliverable here. 
\item \textbf{Easter vacation:} Extensions and writing dissertation main
  chapters.

\item \textbf{Easter term 0--2:}  Further evaluation and complete dissertation.

\item \textbf{Easter term 3:} Proof reading and then an early submission
  so as to concentrate on examination revision.

\end{enumerate}

\end{document}
